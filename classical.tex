\documentclass[12pt]{article}
\begin{document}

\title{Theoretical Minimum: Classical Mechanics}

\maketitle

\section{Admissible Laws}

A physical law constitutes of a phase space $\phi$ and a transition function
$f$.  The physical system evolves as $P_{t+1} = f(P)$ with $P \in \phi$.  The
transition function needs to be:

\begin{itemize}
\item Deterministic -- knowing the state at time $T$ is enough to compute the
  state at an arbitrary time in future.  This follows from $f$ being a
  function.
\item Reversible -- knowing the state at time $T$ is enough to compute the state
  at an arbitrary time in the past.
\end{itemize}

\subsection{Example of inadmissible law}

Aristotle's law of motion $m\dot{\overrightarrow{x}} = \overrightarrow{F}$ is
inadmissible.  To see why this is the case, consider what happens if we attach a
block of mass $1$ to a spring with spring constant $1$.  Then the law of motion
becomes $\dot{x} = -x$, which solves to $x=x_0 e^{-t}$.  This is not reversible
-- both $x_0=0$ and $x_0 = \epsilon$ eventually map to $x_{final}=0$.

\section{Lagrangian mechanics}

The Lagrangian formulation of mechanics state:

\begin{itemize}
\item Every physical system has an associated Lagrangian = $L(q, \dot{q}, t)$.
  For many simple systems $L(q, \dot{q}, t) = K(\dot{q}) - V(q)$ where $K$ is
  the kinetic energy of the system and $V$ is the potential energy of the
  system.  It may not always cleanly separate into kinetic and potential
  energies this way.
\item $P_i$, the generalized momentum with respect to the $i^{th}$ co-ordinate
  is defined to be $\frac{dL}{d\dot{q_i}}$.
\item The Euler-Lagrange equation dictates the laws of motion:
  $$\frac{dP_i}{dt} = \frac{\partial L}{\partial q_i}$$
\end{itemize}
 
\end{document}
