\documentclass[12pt]{article}
\begin{document}

\title{Theoretical Minimum: Classical Mechanics}

\maketitle

\section{Admissible Laws}

A physical law constitutes of a phase space $\phi$ and a transition function
$f$.  The physical system evolves as $P_{t+1} = f(P)$ with $P \in \phi$.  The
transition function needs to be:

\begin{itemize}
\item Deterministic -- knowing the state at time $T$ is enough to compute the
  state at an arbitrary time in future.  This follows from $f$ being a
  function.
\item Reversible -- knowing the state at time $T$ is enough to compute the state
  at an arbitrary time in the past.
\end{itemize}

\subsection{Example of inadmissible law}

Aristotle's law of motion $m\dot{\overrightarrow{x}} = \overrightarrow{F}$ is
inadmissible.  To see why this is the case, consider what happens if we attach a
block of mass $1$ to a spring with spring constant $1$.  Then the law of motion
becomes $\dot{x} = -x$, which solves to $x=x_0 e^{-t}$.  This is not reversible
-- both $x_0=0$ and $x_0 = \epsilon$ eventually map to $x_{final}=0$.

\section{Lagrangian mechanics}

The Lagrangian formulation of mechanics state:

\begin{itemize}
\item Every physical system has an associated Lagrangian = $L(q, \dot{q}, t)$.
  For many simple systems $L(q, \dot{q}, t) = K(\dot{q}) - V(q)$ where $K$ is
  the kinetic energy of the system and $V$ is the potential energy of the
  system.  It may not always cleanly separate into kinetic and potential
  energies this way.
\item $P_i$, the generalized momentum with respect to the $i^{th}$ co-ordinate
  is defined to be $\frac{dL}{d\dot{q_i}}$.
\item The Euler-Lagrange equation dictates the laws of motion:
  $$\frac{dP_i}{dt} = \frac{\partial L}{\partial q_i}$$
\end{itemize}

\section{Symmetries and conservation}

A ``symmetry'' is change in the inputs to the Lagrangian through which the value
of the Lagrangian does not change.

\subsection{Co-ordinate translation symmetry}

If $L$ is invariant under $\partial q_i = f_i(\overrightarrow{q}) \epsilon$ then
the quantity $Q = \sum_{i} P_{i} f_i(\overrightarrow{q})$ is conserved.

\subsection{Time translation symmetry}

Let $H$ (the Hamiltonian) be defined as $H = \sum_i P_i \dot{q}_i$.  Then using
the Euler-Lagrange equations, we can show that $\frac{H}{t} = \frac{\partial
  L}{\partial t}$.  If $L$ does not depend directly on $t$ then we say $L$ has
time-translation symmetry, and $H$ does not change with time.  $H$ is the
definition of energy, so time-translation symmetry implies energy conservation.

\section{Hamiltonian mechanics}

$H$ can be written as a function of $p$ and $q$ by solving for $\dot q$ in term
of $p$.  In cases where we can't solve for $\dot q$ the Hamiltonian is invalid
(disallowed by QM).

Hamilton's equations: $\frac{\partial H}{\partial p_i}=\dot{q_i}$ and
$\frac{\partial H}{\partial q_i}=-\dot{p_i}$.

\end{document}
